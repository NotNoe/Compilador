\documentclass{article}

\usepackage[utf8]{inputenc}
\usepackage{fullpage}
\usepackage{graphicx}
\usepackage[spanish]{babel}
\usepackage{amsfonts}
\usepackage{amssymb}
\usepackage{amsthm}
\usepackage{amsmath}
\usepackage[nottoc]{tocbibind} %Adds "References" to the table of contents
\usepackage{multicol}
\usepackage{xcolor}
\usepackage{listings}

\definecolor{mGreen}{rgb}{0,0.6,0}
\definecolor{mGray}{rgb}{0.5,0.5,0.5}
\definecolor{mPurple}{rgb}{0.58,0,0.82}
\definecolor{backgroundColour}{rgb}{0.95,0.95,0.92}

\lstdefinestyle{CStyle}{
    backgroundcolor=\color{backgroundColour},   
    commentstyle=\color{mGreen},
    keywordstyle=\color{magenta},
    numberstyle=\tiny\color{mGray},
    stringstyle=\color{mPurple},
    basicstyle=\footnotesize,
    breakatwhitespace=false,         
    breaklines=true,                 
    captionpos=b,                    
    keepspaces=true,                 
    numbers=left,                    
    numbersep=5pt,                  
    showspaces=false,                
    showstringspaces=false,
    showtabs=false,                  
    tabsize=2,
    language=C
}


\newtheorem{theorem}{Teorema}
\newtheorem{lemma}[theorem]{Lema}
\newtheorem{definition}{Definición}[section]
\newtheorem{example}{Ejemplo}[section]
\title{%
  Definición de la sintaxis del lenguaje C$--$ \\
  \large Procesadores del Lenguaje Entrega 1}

\author{Óscar Calvet Sisó, Noelia Barranco Godoy}
\date{}
\begin{document}

\maketitle

\section{Introducción}
El objetivo de este documento es el de especificar la sintaxis de nuestro lenguaje, C$--$. Este va a ser un lenguaje de programación imperativo y fuertemente tipado. Nos vamos a basar en el padre de todos los lenguajes imperativos modernos: C. Nuestro objetico con C$--$ es el de crear una versión simplificada de C y añadirle algunas utilidades como la definición de clases (sin herencia y declaradas en el ámbito global de nuestro programa). Contaremos únicamente con un archivo de código.

\noindent El punto de entrada de todo programa válido de nuestro lenguaje será una función 'main', que no devolverá nada y recibirá un array de ints (potencialmente vacío).

\noindent Nuestro lenguaje implementará comentarios de una sola linea al estilo de C. Cabe destacar que la cadena // estará prohibida dentro de los comentarios. Por último, al final expondremos una lista de palabras reservadas. Estas son palabras que por su construcción bien podrían ser identificadores, pero nosotros las prohibiremos porque se usarán en otros contextos.

\section{Identificadores y ámbitos de definición}
El lenguaje permite declarar variables (o constantes) simples y arrays de cualquiera de los tipos descritos más adelante (incluidos arrays de varias dimensiones). Estas variables podrán estar alojadas en memoria estática o dinámica. Como es habitual, habrá anidamiento de bloques con sus respectivos y funciones, que podrán tener argumentos de cualquier tipo.

\noindent Las funciones solo se podrán definir en el ámbito más general del proyecto (fuera de cualquier función), así como, los tipos de usuario, tipos estructurados y clases. Las funciones podrán recibir como argumentos parámentros de cualquier tipo, incluido punteros o referencias.

\noindent En el ámbito global, también se podrán declarar variables, definir tipos nuevos .

\section{Tipos}

\subsection{Tipos básicos y de usuario}

En nuestro lenguaje existirán tres tipos básicos, el tipo entero (con identificador int), el tipo booleano (con identificador bool) y el tipo caracter (con identificador char). Además, el usuario podrá declarar pseudónimos de cualquier tipo del lenguaje así como, clases y estructurados.

\subsection{Arrays y punteros}
Con todos estos tipos, podremos crear arrays y punteros. La notación que identifica a los arrays será la habitual, es decir, el identificador seguido de tantos corchetes de apertura y cierre como dimensiones tenga el array. Los arrays no serán polimórficos.

La notación pertinente a los punteros serán el identificador precedido de la palabra reservada 'pointer' seguida de otra identificador de tipo. Añadiremos pues un operador de referencia que, dada una variable cualquiera devuelva un puntero que apunte a esa variable, que se expresará con el "\&".

\subsection{Funciones}

A nivel sintáctico, todas las funciones estarán tipadas. Para esto exisistirá la palabra reservada "void", que indicará que una función no devuelve nada. No obstante, void no será un tipo realmente (por lo que no podrá usarse para declarar variables), void será una manera de indicar al compilador que esa función no va a devoler nada (realmente será un procedimiento).

\subsection{Declaraciones}

Todo tipo que no sea una función se declará de la siguiente manera: identificador de tipo seguido de un identificador (no seguido de un paréntesis para diferenciarlo de una función). Adicionalmente, podrá ir seguido del signo igual y de una expresión del tipo a declarar. 

\noindent Los tipos de array se declararán de la forma una expresión de tipo existente seguida del símbolo [, una expresión y ]. Los tipos punteros se definirán con la palabra reservada 'pointer' seguida de un tipo cualquiera y terminada en un identificador seguido de un número arbitrario de corchetes cada uno conteniendo una expresión.

\noindent Las funciones se declararán de la siguiente forma: un identificador de tipo (o identificador void), seguido de un identificador y de un parentesis. Una sucesión de declaración de variables separadas por comas, terminado por un paréntesis de cierre y seguido de un bloque de código de nuestro lenguaje.

\noindent Los tipos estructurados se declarán en el ámbito general, serán inicializados por la palabra reservada "struct"  seguidos de llaves que contendran declariones de variables y a continuación un identificador terminado en ';'.

\noindent Las clases se declarán en el ámbito general, serán inicializadas por la palabra reservada class seguida de llaves que contendrán instrucciones, declaraciones y declaraciones de funciones, después de las llaves aparecerá un identificador terminado en ';'.

\section{Conjunto de instrucciones del lenguaje}

\subsection{Designadores de variable}
Se considerará un designador de variable cualquier cadena alfanumérica o con el simbolo "\_" que no empiece por un número y también se considerará designador de variable otro designador de variable, seguido de un corchete de apertura, un número y un corchete de cierre. Esto es para considerar los accesos a arrays como designadores de variables.

\subsection{Asignaciones}
Toda asignación será una construcción de la forma o bien un identificador de variable, o bien una declaración de variable, seguida de un simbolo = y por último una expresión.

\subsection{Llamada a función}
Una instrucción de la forma identificador, paréntesis de apertura, una serie de identificadores de variable o literales separados por comas, y terminado por un cierre de paréntesis.
En cualquier lugar donde pudiera ir un literal de cualquier tipo, también podrá aparecer una llamada de función en su lugar. En lugar de un identificador como nombre de función, podrá usarse la palabra reservada build, que tendrá la función de constructor en una clase (cuando le demos semántica).

\subsection{Expresiones aritméticas}

Siguiendo sintaxis habitual para estos, son un conjunto formado por $+,-,*,/, \%$ como operadores binarios aplicados sobre otras expresiones aritméticas. Unos identificadores constantes (números) e identificadores de variables.

También el operador prefijo $"-"$ seguido de otra expresión aritmética, una operación aritmética rodeada de paréntesis.

\subsection{Expresiones booleanas}
Otra vez siguiendo el estándar, un operador booleano se definirá con los casos base de palabras reservadas "true" y "false", y la aplicación de los operadores binarios $"\&\&, \|"$ de manera infija sobre otras expresiones booleanas o el operador unario prefijo "!".
También se considerarán expresiones booleanas las que contienen dos identificadores de variable o expresiones de cualquier tipo y los siguientes símbolos $"==, !="$ separándolas como operador infijo, representando la igualdad y la desigualdad respectivamente.

Finalmente también existirán expresiones booleanas de la forma siguiente: dos expresiones aritméticas separadas por los operadores "$<=, >=, >, <$" de manera infija, representado las comparaciones aritméticas.



\subsection{Expresiones de caracteres}
Estas solamente serán de la forma comillas simples, seguida de un caracter ASCII y cerrado por otra comilla.

\subsection{Expresiones de arrays}
Al definir un array, siempre se hará de manera constante, osea la apertura de un corchete seguida de uan secuencia de literales separados por coma y terminado en corchete de cierre


\subsection{Expresiones con punteros}

Tenemos el operadores unarios "$*,\&$" precede a un identificador, para acceder al valor de un puntero o a su dirección respectivamente.
También tenemos el operador binario infijo "." que se encontrará entre dos identificadores donde el de más a la izquierda hará referencia a un struct o una clase y el de la derecha a un identificador de variable o de función. La prioridad de estos operadores será la habitual de C.


\subsection{Instrucción}
Toda instrucción será una asignación, una de las instrucciones definidas a continuación. En el caso en el que sea una instrucción de tipo if, while o for, no terminará en punto y coma, pero si la instrucción es de tipo asignación o de cualquier otro tipo, terminará en punto y coma.


\subsection{Intrucción while}

Para representar bucles condicionales utilizaremos la siguiente sintaxis: la palabra reservada "while" seguida de paréntesis que contendrán una expresión (deberá ser booleana) y seguido de llaves que contendrán instrucciones.

\subsection{Instrucción for}

Para representar bucles con contadores utilizaremos la suguiente sintaxis: la palabra reservada "for" seguida de paréntesis que contendrán exactamente 3 instrucciones (declaración de variable, expresión booleana y una instrucción) seguido de llaves que contendrán instrucciones. Se separarán entre sí usando ';' y la última también habrá de terminar en este caracter.

\subsection{Instrucción if simple}

Para la instrucción condicional simple utilizaremos la sintaxis siguiente: la palabra reservada if seguida de paréntesis que contendrán una expresión (será de tipo booleana) y a continuación llaves que contendrán instrucciones.

\subsection{Instrucción if compuesta}

Para la instrucción condicional compuesta utilizaremos la sintaxis siguiente: la palabra reservada if seguida de paréntesis que contendrán una expresión (será de tipo booleana) y a continuación llaves que contendrán instrucciones. Seguidamente aparecerá la palabra reservada else seguida de llaves que contendrán instrucciones.

\subsection{Instrucción switch}

Para la instrucción switch utilizaremos la siguiente sintaxis: la palabra reservada switch seguida de paréntesis que contendrán un identificador seguido de llaves que contendrán bloques de cases consistentes del la palabra reservada case seguida de un identicador seguido de código entre llaves. Existe un bloque case especial que en lugar de empezar por la palabra case empieza por la palabra default.

\subsection{Instrucción break}

La instrucción break tendrá la sintaxis de siguiente: la palabra 'break'. Se utilizará para salir de los bucles.

\subsection{Instrucción continue}

La instrucción continue tendrá la sintaxis de siguiente: la palabra 'continue'. Se utilizará para saltar a la siguiente iteración en los bucles.

\subsection{Instrucción return}

La instrucción return tendrá la siguiente sintaxis: la palabra reservada 'return' seguida de una expresión. Se utilizará argumento de retorno de una función.

\subsection{Instrucción new}

La instrucción new tendrá la siguiente sintaxis: la palabra reservada 'new' seguida de un identificador de tipo simple o compuesto. Podrá estar seguido de una o más aperturas de corchete con un número entre ellos, esto indicará que son arrays. Se utilizará para reservar memoria dinámica.

\subsection{Instrucción delete}

La instrucción delete tendrá la siguiente sintaxis: la palabra reservada 'delete' seguida de un identificador. Se utilizará para eliminar memoria dinámica.

\subsection{Definición de tipos de usuario}
La palabra reservada 'typedef' se usará en el ámbito global para definir alias de cualquier tipo.

\subsection{Identificador this}

La palabra reservada 'this' comporta como un identificador de clase y podrá aparecer allá donde aparecen estos. Luego a nivel semántico manejaremos qué pasa si la utilizamos en un contexto inadecuado.


\section{Gestión de errores}
La gestión de errores va a indicar el tipo de error, la ubicación en el fichero de código fuente (fila y colunma) y parará la compilación. Se intentará proseguir la compilación para detectar más errores.

\section{Ejemplos}


\begin{lstlisting}[style=CStyle]
void main(){
    int a = 5;
    int b[3] = [1, 2, 3];
    pointer int c = new int;
    *c = 5 + (7 % 2);
    b = *c + a;
    pointer int hola = new int[5];
    for(int i = 0; i < 27 + b[2]; i = i + 1;){
        if(b[i] % 2 == 0){
            continue;
        }
        else{
            break;
        }
     }
     int i = 0;
     bool finished = false;
     while(i <= 2 && !finished){
        delete hola[i];
        i = i + 1;
     }
}
\end{lstlisting}



\begin{lstlisting}[style=CStyle]
//Esto es un comentario
struct{
    int a;
    int b = 2;
}tStruct;

class{
    int a;

    tClase build(int a){
        this.a = a;
    }

    void protocolo(int a){}
    int function() {
        this.a = this.a + 1;
        return this.a;
    }

}tClase;

void idonthaveatype(bool &b){ 
    b = false;
}

int ihaveatype(pointer int k, int j){
    *k = j + 1;
    return *k;
}

void main(){
    tClase ejemplo = tClase(5);
    ejemplo.protocolo(2);
    int a = ejemplo.funcion();
    bool c = true;
    idonthaveatype(c);
    ihaveatype(&a, 5);
}

void main(){
    tClase ejemplo = tClase(5);
    ejemplo.protocolo(2);
    int a = ejemplo.funcion();
    bool c = true;
    idonthaveatype(c);
    ihaveatype(&a, 5);
}
\end{lstlisting}

\begin{lstlisting}[style=CStyle]
typedef int aux;

void main(){
    int a[2] = [5, 2];
    switch(a){
    case 1 {
        int c = 3;
        break;
    }
    case 2 {
        if(a == 2){
            aux b = 3;
        }
    }
    default {
        int d = 3;
    }
    } 
}
\end{lstlisting}

\section{Palabras reservadas}
\begin{multicols}{4}
\begin{itemize}
    \item int
    \item char
    \item bool
    \item class
    \item struct
    \item for
    \item while
    \item if
    \item switch
    \item break
    \item continue
    \item const
    \item void
    \item default
    \item case
    \item new
    \item return
    \item true
    \item false
    \item this
    \item delete
    \item typedef
    \item pointer

\end{itemize}
\end{multicols}

\end{document}
